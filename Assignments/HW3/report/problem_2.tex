\newpage
\section*{Problem 2:}
\begin{enumerate}
\item If $A \in \mathbb{R}^{n\times n}$ is a skew-symmetric, then $x^{T}Ax =0$ since the diagonal elements of the $A$ by definition are zero. We can show that for $A^{2j+1}$ for $j=0,1,2, \dots$ (i.e., raising $A$ to an odd power) will result into a skew-symmetric. 

For a skew-symmetric matrix $A$ (i.e., $A^{T} = -A$), we have 

\begin{align*}
& A^{m} = (-A^{T})^{m}  \\
& A.A.A\dots=(-1)^{m}.A^{T}.A^{T}.A^{T}\dots =(-1)^{m}A^{m} \\
\end{align*}


Thus, if $m$ is odd, then $A^{m} = -(A^{T})^{m} = -(A^{m})^{T}$ and thus the resulting matrix is skew-symmetric. Since any skew-symmetric matrix has zero diagonal elements, we can deduce that 
$$
x^{T}A^{2j+1}x = 0\quad \forall j=0,1,2\dots \text{ and } x\in \mathbb{R}^{n}
$$

\item Find the eigenvalues of $A$ can be done by solving the following for $\lambda$

\begin{align*}
&Ax = \lambda x\\
\end{align*}

From (a), we can multiply the above by $x^{T}$ to get 

\begin{align*}
& x^{T}Ax = \lambda x = x^{T}\lambda x = \lambda \parallel x \parallel^{2} = 0\\
\end{align*}

One solution is $\lambda = 0$. Since $x$ is a (non-trivial) eigenvector, then $\parallel x \parallel^{2} \neq 0$. However, if we consider $\lambda$ as an imaginary, then we can write the above as

\begin{align*}
& (\lambda_{RE} + i \lambda_{IM})  \parallel x \parallel^{2} = 0\\
& -\lambda_{RE}\parallel x \parallel^{2} = i \lambda_{IM}\parallel x \parallel^{2}\\ 
& \left(\lambda_{RE}\right)^{2} = (-1) \left(\lambda_{IM}\right)^{2}\\ 
\end{align*}

where $\lambda_{RE}$ is the real part and $\lambda_{IM}$ is the imaginary part of $\lambda$.  Besides the zero eigenvalue, the above shows that the eigenvalues $\lambda$ are purely imaginary since $\lambda_{IM}$ can not be zero. Also, the eigenvalues comes in as conjugate pairs since $A$ is a real matrix. 

From the lectures notes, we have the following fact about Krylov subspace. If $A$ is diagonalizable, then $d(A,r_{0})$ is the minimum number of eigenvectors in the eigendecomposition of $A$. From (b), since the eigenvalues comes in conjugate pairs, then the number of eigenvectors is even and thus $d(A,r_{0})$ is even. 

Now we need to prove that $A$ is diagonalizable. Since $A$ is a normal matrix i.e., $A^{T}A = AA^{T} = -AA $, then it must be diagonalizable (following the Spectral theorem). 

\item ??????????
\item ??????????



\end{enumerate}