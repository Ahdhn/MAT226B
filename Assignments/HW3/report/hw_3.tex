\documentclass[12pt] {article}
\usepackage{times}
\usepackage[margin=1in,bottom=1in,top=0.6in]{geometry}

\usepackage{hhline}
\usepackage{subfig}
\usepackage{amsmath}
\usepackage{amsfonts}
\usepackage[inline,shortlabels]{enumitem}%enumerate with letters
\usepackage{mathrsfs} 
\usepackage[square,numbers]{natbib}
\usepackage{graphicx}
\bibliographystyle{unsrtnat}
\usepackage{float}
\usepackage[dvipsnames]{xcolor}
\usepackage{tikz}
\usepackage{amsfonts,amsmath, color, fullpage, graphicx, mathtools, empheq, amsthm, amssymb}
\usepackage{wasysym}
\usepackage{thmtools}
\usepackage{listings}
\usepackage[ruled,vlined, linesnumbered]{algorithm2e}
\usepackage{flexisym}
\usepackage{breqn}
\setenumerate[0]{label=(\alph*)}

\usepackage[framed,numbered,autolinebreaks,useliterate]{../../mcode}

%To write numbers in scientific notation nicely
%use as \expnumber{2.000000000000000}{+00} to write 2.000000000000000e+00
\newcommand{\expnumber}[2]{{#1}\mathrm{e}{#2}}



\begin{document}

\title{MAT 226B – Large Scale Matrix Computation \\ Homework 3}
\author{Ahmed Mahmoud}
\date{February, 27th 2020} 

\maketitle




%============Table========
%\begin{figure}[tbh]
% \centering    
%\begin{tabular}{ |p{4cm}|| p{2cm}|p{2cm}|p{2cm}|p{2cm}|}
% \hline
% & Processor 1 &  Processor 2  & Processor 3 & Processor 4\\ \hhline{|=|=|=|=|=|}
% \hline
% Performance          &$1.08$        &$1.425$       &\textbf{1.52}  &   \\
% \hline
%\end{tabular} 
%\caption{Metric table for the four processors}
%   \label{tab:metric}
%\end{figure} 
%============Figure========
%\begin{figure}[!tbh]
%\centering        
%   \subfloat {\includegraphics[width=0.65\textwidth]{fig2_4.png}}
%   \caption{ }
%   \label{fig:fig}
%\end{figure}
%============Code========
%\begin{lstlisting}
%function [f,g] = linear_regression(theta, X,y)
%end
%\end{lstlisting}



\newpage
\section*{Problem 1:}

\section*{Problem 2:}
Here we are required to find an efficient way to compute $q = Mv$ and $q = M^{T}v$ for $v \in \mathbb{C}^{n}$ where $M = (A - s_{0}E)^{-1}E$. We can compute the matrix-vector multiplication efficiently using LU factorization. We first can write the multiplication as

\begin{align*}
& q = (A-s_{0}E)^{-1}Ev = (\underbrace{A-s_{0}E}_{W})^{-1}\underbrace{Ev}_{f} \\
& q = W^{-1} f \quad \Rightarrow \quad Wq = f \quad \Rightarrow \quad \underbrace{PD^{-1}WQ}_{LU}\underbrace{Q^{T}q}_{d} = PD^{-1}f \\
\end{align*}
Thus, we can fist solve $Lc = PD^{-1}f$ for $c \in \mathbb{C}^{n}$ via forward substitution, then solve $Ud = c$ for $d \in \mathbb{C}^{n}$ via backward substitution, and finally set $q = Qd$. 

We can use the same LU factorization to compute $q = M^{T}v$ efficiently. We first not that transposing the LU factorization for a given matrix $W$ is $U^{T}L^{T} = Q^{T}W^{T}D^{-T}P^{T}$
 We can write this multiplication as 

\begin{align*}
& q = ((A-s_{0}E)^{-1}E)^{T}v = E^{T}\underbrace{(A-s_{0}E)^{-T}v}_{g}\\
& g = W^{-T} v \quad \Rightarrow \quad W^{T}g = v \quad \Rightarrow \quad \underbrace{Q^{T}W^{T}D^{-T}P^{T}}_{U^{T}L^{T}}\underbrace{(D^{-T}P^{T})^{-1}g}_{d}=Q^{T}v\\
\end{align*}

Thus, we can first solve $U^{T}c = Q^{T}v$ for $c$ via forward substitution, then solve $L^{T}d = c$ for $d$ via backward substitution, and then set $g = D^{-T}P^{T}d$. Finally, we multiply $g$ from the left by $E^{T}$ to get $q$. The functions \texttt{Mv} and \texttt{transposeMv} implements these operations as discussed. 


\newpage
\section*{Problem 3:}
\begin{enumerate}
\item Function \texttt{hermitian\_lanczos} implements the Hermitian Lanczos process where it takes the matrix $A$, vector $r$, and \texttt{KMAX} parameter and output $T_{k}$ tridiagonal matrix in sparse format. We used this function to approximate eigenvalues of $A$ from \texttt{make\_3d\_laplacian} function using $k=7$. Table~\ref{tab:app_eig_7} shows the results where the first two columns show the exact eigenvalues of $A$ (computed from the provided function) along its multiplicity and last column shows the approximate eigenvalues. We can see that the approximate solution is able to capture all the eigenvalues with very high accuracy.

\begin{figure}[tbh]
 \centering    
\begin{tabular}{ |p{5cm}| p{2cm}|| p{5cm}|}
\hline
 Exact Eigenvalues  & Multiplicity & Approximate Eigenvalues \\ \hhline{|=|=|=|}   
$\expnumber{1.757359312880715}{+00}$ & 1& $\expnumber{1.757359312880715}{+00}$ \\
$\expnumber{3.171572875253810}{+00}$ &3 & $\expnumber{3.171572875253810}{+00}$ \\
$\expnumber{4.585786437626905}{+00}$ &6 & $\expnumber{4.585786437626905}{+00}$ \\
$\expnumber{6.000000000000000}{+00}$ & 7& $\expnumber{6.000000000000002}{+00}$ \\
$\expnumber{7.414213562373095}{+00}$&6 & $\expnumber{7.414213562373095}{+00}$ \\
$\expnumber{8.828427124746190}{+00}$ &3 & $\expnumber{8.828427124746192}{+00}$ \\
$\expnumber{1.024264068711928}{+01}$ &1 & $\expnumber{1.024264068711929}{+01}$ \\
\hline
\end{tabular} 
\caption{Exact and approximate eigenvalues using Hermitian Lanczos process}
   \label{tab:app_eig_7}
\end{figure}

\item We used \texttt{hermitian\_lanczos} to compute the approximate eigenvalues for the 262144 \times 262144 matrix from \texttt{make\_3d\_laplacian(64)}. Tables~\ref{tab:app_eig_100,tab:app_eig_200,tab:app_eig_400,tab:app_eig_800,tab:app_eig_1200, tab:app_eig_1600,tab:app_eig_2000} show the 10 smallest and 10 largest approximated eigenvalues. Table~\ref{tab:exact_64} shows the 10 smallest and 10 largest exact eigenvalues. 

\begin{figure}[tbh]
 \centering    
\begin{tabular}{ ||p{6cm}||p{6cm}|}
\hline
 Smallest Eigenvalues & Largest Eigenvalues \\ \hhline{|=|=|}   
\hline
$\expnumber{1.422787575075167}{-02}$ & $\expnumber{1.170222368514153}{+01}$ \\  
$\expnumber{2.522292134967839}{-02}$ & $\expnumber{1.175506886438500}{+01}$ \\  
$\expnumber{3.720946521956244}{-02}$ & $\expnumber{1.180683457939747}{+01}$ \\  
$\expnumber{6.015341020592301}{-02}$ & $\expnumber{1.184896392339356}{+01}$ \\  
$\expnumber{8.407846117839080}{-02}$ & $\expnumber{1.188540931980204}{+01}$ \\  
$\expnumber{1.168498457907536}{-01}$ & $\expnumber{1.191749739736500}{+01}$ \\  
$\expnumber{1.581734700307590}{-01}$ & $\expnumber{1.194638621492721}{+01}$ \\  
$\expnumber{1.995732825474702}{-01}$ & $\expnumber{1.196512279467138}{+01}$ \\  
$\expnumber{2.466691832816189}{-01}$ & $\expnumber{1.198055273500129}{+01}$ \\  
$\expnumber{3.011244810109077}{-01}$ & $\expnumber{1.199293004521624}{+01}$ \\  
\hline  
\end{tabular} 
\caption{Approximate Eigenvalues using $K = 100$}
   \label{tab:app_eig_100}
\end{figure} 


\begin{figure}[tbh]
 \centering    
\begin{tabular}{ ||p{6cm}||p{6cm}|}
\hline
 Smallest Eigenvalues & Largest Eigenvalues \\ \hhline{|=|=|}   
\hline
$\expnumber{7.006926067695284}{-03}$ & $\expnumber{1.191194541966088}{+01}$ \\  
$\expnumber{1.400816296330363}{-02}$ & $\expnumber{1.192367820060319}{+01}$ \\  
$\expnumber{2.104749654401042}{-02}$ & $\expnumber{1.193815033138952}{+01}$ \\  
$\expnumber{2.644416419643677}{-02}$ & $\expnumber{1.194943725586547}{+01}$ \\  
$\expnumber{3.252921177956383}{-02}$ & $\expnumber{1.195823115974695}{+01}$ \\  
$\expnumber{4.248144987518579}{-02}$ & $\expnumber{1.196722905407029}{+01}$ \\  
$\expnumber{5.158180541911261}{-02}$ & $\expnumber{1.197279443138257}{+01}$ \\  
$\expnumber{6.175958581860198}{-02}$ & $\expnumber{1.197897280487774}{+01}$ \\  
$\expnumber{7.373377938324976}{-02}$ & $\expnumber{1.198599186623671}{+01}$ \\  
$\expnumber{8.527634111033579}{-02}$ & $\expnumber{1.199299336087879}{+01}$ \\  
\hline  
\end{tabular} 
\caption{Approximate Eigenvalues using $K = 200$}
   \label{tab:app_eig_200}
\end{figure} 

\begin{figure}[tbh]
 \centering    
\begin{tabular}{ ||p{6cm}||p{6cm}|}
\hline
 Smallest Eigenvalues & Largest Eigenvalues \\ \hhline{|=|=|}    
\hline
$\expnumber{7.006639006043782}{-03}$ & $\expnumber{1.195105825040107}{+01}$ \\  
$\expnumber{1.400782323540659}{-02}$ & $\expnumber{1.195568924396476}{+01}$ \\  
$\expnumber{2.100900746475762}{-02}$ & $\expnumber{1.195806895929420}{+01}$ \\  
$\expnumber{2.565829376859278}{-02}$ & $\expnumber{1.196033932970904}{+01}$ \\  
$\expnumber{2.801019169415529}{-02}$ & $\expnumber{1.196734052200119}{+01}$ \\  
$\expnumber{3.265947799840193}{-02}$ & $\expnumber{1.197198980830585}{+01}$ \\  
$\expnumber{3.966069944862968}{-02}$ & $\expnumber{1.197434170623140}{+01}$ \\  
$\expnumber{4.193108831261764}{-02}$ & $\expnumber{1.197899099253526}{+01}$ \\  
$\expnumber{4.431079048315387}{-02}$ & $\expnumber{1.198599217676459}{+01}$ \\  
$\expnumber{4.897191625334762}{-02}$ & $\expnumber{1.199299336099394}{+01}$ \\  
\hline  
\end{tabular} 
\caption{Approximate Eigenvalues using $K = 400$}
   \label{tab:app_eig_400}
\end{figure} 

\begin{figure}[tbh]
 \centering    
\begin{tabular}{ ||p{6cm}||p{6cm}|}
\hline
 Smallest Eigenvalues & Largest Eigenvalues \\ \hhline{|=|=|}    
\hline
$\expnumber{7.006639006040011}{-03}$ & $\expnumber{1.197198980830589}{+01}$ \\  
$\expnumber{7.006639006045063}{-03}$ & $\expnumber{1.197434170555307}{+01}$ \\  
$\expnumber{1.400782323539451}{-02}$ & $\expnumber{1.197434170623140}{+01}$ \\  
$\expnumber{1.400782323540438}{-02}$ & $\expnumber{1.197899099253525}{+01}$ \\  
$\expnumber{2.100900746475027}{-02}$ & $\expnumber{1.197899099253527}{+01}$ \\  
$\expnumber{2.100900746475296}{-02}$ & $\expnumber{1.198599217676461}{+01}$ \\  
$\expnumber{2.565829371816093}{-02}$ & $\expnumber{1.198599217676461}{+01}$ \\  
$\expnumber{2.565829376859185}{-02}$ & $\expnumber{1.199293916949044}{+01}$ \\  
$\expnumber{2.801018851772900}{-02}$ & $\expnumber{1.199299336099396}{+01}$ \\  
$\expnumber{2.801019169410766}{-02}$ & $\expnumber{1.199299336099397}{+01}$ \\  
\hline  
\end{tabular} 
\caption{Approximate Eigenvalues using $K = 800$}
   \label{tab:app_eig_800}
\end{figure} 

\begin{figure}[tbh]
 \centering    
\begin{tabular}{ ||p{6cm}||p{6cm}|}
\hline
 Smallest Eigenvalues & Largest Eigenvalues \\ \hhline{|=|=|}   
\hline
$\expnumber{7.006639006034983}{-03}$ & $\expnumber{1.197899099253524}{+01}$ \\  
$\expnumber{7.006639006042774}{-03}$ & $\expnumber{1.197899099253525}{+01}$ \\  
$\expnumber{7.006639006050910}{-03}$ & $\expnumber{1.197899099253525}{+01}$ \\  
$\expnumber{7.006642433917978}{-03}$ & $\expnumber{1.198599217676460}{+01}$ \\  
$\expnumber{1.400782323538701}{-02}$ & $\expnumber{1.198599217676461}{+01}$ \\  
$\expnumber{1.400782323539747}{-02}$ & $\expnumber{1.198599217676462}{+01}$ \\  
$\expnumber{1.400782323539999}{-02}$ & $\expnumber{1.199299336099394}{+01}$ \\  
$\expnumber{2.100900746475157}{-02}$ & $\expnumber{1.199299336099396}{+01}$ \\  
$\expnumber{2.100900746475692}{-02}$ & $\expnumber{1.199299336099397}{+01}$ \\  
$\expnumber{2.100900746476310}{-02}$ & $\expnumber{1.199299336099399}{+01}$ \\  
\hline  
\end{tabular} 
\caption{Approximate Eigenvalues using $K = 1200$}
   \label{tab:app_eig_1200}
\end{figure} 

\begin{figure}[tbh]
 \centering    
\begin{tabular}{ ||p{6cm}||p{6cm}|}
\hline
 Smallest Eigenvalues & Largest Eigenvalues \\ \hhline{|=|=|}     
\hline
$\expnumber{7.006639006034724}{-03}$ & $\expnumber{1.197899099253525}{+01}$ \\  
$\expnumber{7.006639006037083}{-03}$ & $\expnumber{1.198599217676456}{+01}$ \\  
$\expnumber{7.006639006037985}{-03}$ & $\expnumber{1.198599217676460}{+01}$ \\  
$\expnumber{7.006639006044512}{-03}$ & $\expnumber{1.198599217676460}{+01}$ \\  
$\expnumber{7.006639006047808}{-03}$ & $\expnumber{1.198599217676461}{+01}$ \\  
$\expnumber{1.400782323539137}{-02}$ & $\expnumber{1.199299336099392}{+01}$ \\  
$\expnumber{1.400782323539370}{-02}$ & $\expnumber{1.199299336099395}{+01}$ \\  
$\expnumber{1.400782323539474}{-02}$ & $\expnumber{1.199299336099396}{+01}$ \\  
$\expnumber{1.400782323540451}{-02}$ & $\expnumber{1.199299336099397}{+01}$ \\  
$\expnumber{1.412781215190661}{-02}$ & $\expnumber{1.199299336099398}{+01}$ \\  
\hline  
\end{tabular} 
\caption{Approximate Eigenvalues using $K = 1600$}
   \label{tab:app_eig_1600}
\end{figure} 

\begin{figure}[tbh]
 \centering    
\begin{tabular}{ ||p{6cm}||p{6cm}|}
\hline
 Smallest Eigenvalues & Largest Eigenvalues \\ \hhline{|=|=|}    
\hline
$\expnumber{7.006639006032009}{-03}$ & $\expnumber{1.198599217676460}{+01}$ \\  
$\expnumber{7.006639006034958}{-03}$ & $\expnumber{1.198599217676460}{+01}$ \\  
$\expnumber{7.006639006035550}{-03}$ & $\expnumber{1.198599217676461}{+01}$ \\  
$\expnumber{7.006639006037528}{-03}$ & $\expnumber{1.198599260901220}{+01}$ \\  
$\expnumber{7.006639006037725}{-03}$ & $\expnumber{1.199299336099394}{+01}$ \\  
$\expnumber{7.006639006042897}{-03}$ & $\expnumber{1.199299336099395}{+01}$ \\  
$\expnumber{1.400782323536894}{-02}$ & $\expnumber{1.199299336099396}{+01}$ \\  
$\expnumber{1.400782323539092}{-02}$ & $\expnumber{1.199299336099396}{+01}$ \\  
$\expnumber{1.400782323539872}{-02}$ & $\expnumber{1.199299336099396}{+01}$ \\  
$\expnumber{1.400782323540259}{-02}$ & $\expnumber{1.199299336099397}{+01}$ \\  
\hline  
\end{tabular} 
\caption{Approximate Eigenvalues using $K = 2000$}
   \label{tab:app_eig_2000}
\end{figure} 


\begin{figure}[tbh]
 \centering    
\begin{tabular}{ ||p{6cm}||p{6cm}|}
\hline
 Smallest Eigenvalues & Largest Eigenvalues \\ \hhline{|=|=|}    
\hline
$\expnumber{7.006639006040594}{-03}$ & $\expnumber{1.197434170623141}{+01}$ \\  
$\expnumber{1.400782323539662}{-02}$ & $\expnumber{1.197434170623141}{+01}$ \\  
$\expnumber{1.400782323539662}{-02}$ & $\expnumber{1.197434170623141}{+01}$ \\  
$\expnumber{1.400782323539662}{-02}$ & $\expnumber{1.197899099253525}{+01}$ \\  
$\expnumber{2.100900746475265}{-02}$ & $\expnumber{1.197899099253525}{+01}$ \\  
$\expnumber{2.100900746475265}{-02}$ & $\expnumber{1.197899099253525}{+01}$ \\  
$\expnumber{2.100900746475265}{-02}$ & $\expnumber{1.198599217676460}{+01}$ \\  
$\expnumber{2.565829376859075}{-02}$ & $\expnumber{1.198599217676460}{+01}$ \\  
$\expnumber{2.565829376859163}{-02}$ & $\expnumber{1.198599217676460}{+01}$ \\  
$\expnumber{2.565829376859163}{-02}$ & $\expnumber{1.199299336099396}{+01}$ \\  
\hline  
\end{tabular} 
\caption{Exact Eigenvalues}
   \label{tab:exact_64}
\end{figure} 

\end{enumerate}




\end{document}
