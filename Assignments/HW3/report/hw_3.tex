\documentclass[12pt] {article}
\usepackage{times}
\usepackage[margin=1in,bottom=1in,top=0.6in]{geometry}

\usepackage{hhline}
\usepackage{subfig}
\usepackage{amsmath}
\usepackage{amsfonts}
\usepackage[inline,shortlabels]{enumitem}%enumerate with letters
\usepackage{mathrsfs} 
\usepackage[square,numbers]{natbib}
\usepackage{graphicx}
\bibliographystyle{unsrtnat}
\usepackage{float}
\usepackage[dvipsnames]{xcolor}
\usepackage{tikz}
\usepackage{amsfonts,amsmath, color, fullpage, graphicx, mathtools, empheq, amsthm, amssymb}
\usepackage{wasysym}
\usepackage{thmtools}
\usepackage{listings}
\usepackage[ruled,vlined, linesnumbered]{algorithm2e}
\usepackage{flexisym}
\usepackage{breqn}
\setenumerate[0]{label=(\alph*)}

\usepackage[framed,numbered,autolinebreaks,useliterate]{../../mcode}

%To write numbers in scientific notation nicely
%use as \expnumber{2.000000000000000}{+00} to write 2.000000000000000e+00
\newcommand{\expnumber}[2]{{#1}\mathrm{e}{#2}}



\begin{document}

\title{MAT 226B – Large Scale Matrix Computation \\ Homework 3}
\author{Ahmed Mahmoud}
\date{February, 27th 2020} 

\maketitle




%============Table========
%\begin{figure}[tbh]
% \centering    
%\begin{tabular}{ |p{4cm}|| p{2cm}|p{2cm}|p{2cm}|p{2cm}|}
% \hline
% & Processor 1 &  Processor 2  & Processor 3 & Processor 4\\ \hhline{|=|=|=|=|=|}
% \hline
% Performance          &$1.08$        &$1.425$       &\textbf{1.52}  &   \\
% \hline
%\end{tabular} 
%\caption{Metric table for the four processors}
%   \label{tab:metric}
%\end{figure} 
%============Figure========
%\begin{figure}[!tbh]
%\centering        
%   \subfloat {\includegraphics[width=0.65\textwidth]{fig2_4.png}}
%   \caption{ }
%   \label{fig:fig}
%\end{figure}
%============Code========
%\begin{lstlisting}
%function [f,g] = linear_regression(theta, X,y)
%end
%\end{lstlisting}



\section*{Problem 1:}
\begin{enumerate}
\item The structure of $A$ looks as following 
$$
A = 
\setcounter{MaxMatrixCols}{11}
\begin{bmatrix}
* & * & 0 & 0 & 0 & 0 & 0 & 0\\
* & 0 & * & 0 & 0 & 0 & 0 & 0\\
* & 0 & 0 & * & 0 & 0 & 0 & 0\\
* & 0 & 0 & 0 & \ddots & 0 & 0 & 0\\
* & 0 & 0 & 0 & 0 & \ddots & 0 & 0\\
* & 0 & 0 & 0 & 0 & 0 & \ddots & 0\\
* & 0 & 0 & 0 & 0 & 0 & 0 & *\\
1 & 0 & 0 & 0 & 0 & 0 & 0 & 0\\
\end{bmatrix}
$$

The Krylov subspace of $K_{k}(A, r_{0}),  \forall k=1,2,\cdots d(A,r_{0})$ where $r_{0} = e_{n}$ is the $n-$th unit vector is 
$$
K_{k}(A, r_{0}) = span\{r_{0}, Ar_{0}, A^{2}r_{0}, \cdots, A^{k-1}r_{0} \}
$$
which can be determined using the following observation. Since $r_{0} = e_{n}$, then $Ar_{0} = \alpha_{n-1}e_{n-1}$, $Ae_{n-1} = \alpha_{n-1}e_{n-2}$, and so on where $\alpha \in \mathbb{R}$ is some factor depends on the non-zero values in $D$. Thus, $K_{k}(A,r_{0}) = span\{e_{1}, e_{2}, \cdots, e_{k}\}$.

To show that $d(A,r_{0}) = n$, we use the following observation that $Ae_{1} = \alpha e_{1}$. Thus, after the first $n$ unit vector, the vectors produced by multiplication by $A$ are no longer linearly independent and thus the $d(A, r_{0})=n$

\item In exact arithmetic, the number of iteration needed by MR method with starting residual vector $r_{0}$ is $n$. We proved in the lecture that $x^{*} = A^{-1}b \in x_{0}+ K_{d}(A,r_{0})$ and $d = d(A,r_{0})$ is the minimum such value. Thus, the number of iterations of MR can not be less than $n$. 

\item 
The sparsity structure of $A^{T}$
$$
A^{T} = 
\setcounter{MaxMatrixCols}{11}
\begin{bmatrix}
* & * & * & \cdots & \cdots & \cdots & * & 1\\
* & 0 & 0 & 0 & 0 & 0 & 0 & 0\\
0 & * & 0 & 0 & 0 & 0 & 0 & 0\\
0 & 0 & \ddots & 0 & 0 & 0 & 0 & 0\\
0 & 0 & 0 & \ddots & 0 & 0 & 0 & 0\\
0 & 0 & 0 & 0 & \ddots & 0 & 0 & 0\\
0 & 0 & 0 & 0 & 0 & * & 0 & 0\\
0 & 0 & 0 & 0 & 0 & 0 & * & 0\\
\end{bmatrix}
$$
 And the sparsity structure of $A^{T}A$ 
$$
A^{T}A = 
\setcounter{MaxMatrixCols}{11}
\begin{bmatrix}
* & * & * & \cdots & \cdots & \cdots & * & *\\
* & * & 0 & 0 & 0 & 0 & 0 & 0\\
* & 0 & * & 0 & 0 & 0 & 0 & 0\\
\vdots & 0 & 0 & \ddots & 0 & 0 & 0 & 0\\
\vdots & 0 & 0 & 0 & \ddots & 0 & 0 & 0\\
\vdots & 0 & 0 & 0 & 0 & \ddots & 0 & 0\\
* & 0 & 0 & 0 & 0 & 0 & * & 0\\
* & 0 & 0 & 0 & 0 & 0 & 0 & *\\
\end{bmatrix}
= 
$$

$$
\setcounter{MaxMatrixCols}{11}
\begin{bmatrix}
* & * & * & \cdots & \cdots & \cdots & * & *\\
* & 0 & 0 & 0 & 0 & 0 & 0 & 0\\
* & 0 & 0 & 0 & 0 & 0 & 0 & 0\\
\vdots & 0 & 0 & 0 & 0 & 0 & 0 & 0\\
\vdots & 0 & 0 & 0 & 0 & 0 & 0 & 0\\
\vdots & 0 & 0 & 0 & 0 & 0 & 0 & 0\\
* & 0 & 0 & 0 & 0 & 0 & 0 & 0\\
* & 0 & 0 & 0 & 0 & 0 & 0 & 0\\
\end{bmatrix} + 
\setcounter{MaxMatrixCols}{11}
\begin{bmatrix}
* & 0 & 0 & 0 & 0 & 0 & 0 & 0\\
0 & * & 0 & 0 & 0 & 0 & 0 & 0\\
0 & 0 & * & 0 & 0 & 0 & 0 & 0\\
0 & 0 & 0 & \ddots & 0 & 0 & 0 & 0\\
0 & 0 & 0 & 0 & \ddots & 0 & 0 & 0\\
0 & 0 & 0 & 0 & 0 & \ddots & 0 & 0\\
0 & 0 & 0 & 0 & 0 & 0 & * & 0\\
0 & 0 & 0 & 0 & 0 & 0 & 0 & *\\
\end{bmatrix}
$$
The first matrix is a rank 2 matrix which has at most two distinct eigenvalue while the second matrix is a diagonal (identity) matrix that has only one distinct eigenvalue. Thus, $A^{T}A$ can have at most three eigenvalues. 

\item 
\end{enumerate}
\newpage
\section*{Problem 2:}
\begin{enumerate}
\item For the given matrix $U^{(k)}$, there will be \textbf{10} fill-in elements as shown in the following matrix if the $k$-th step of sparse LU factorization is performed without pivoting where the fill-in elements are shown with $\color{red}{\textbf{+}}$

$$
U^{(k)} = 
\setcounter{MaxMatrixCols}{11}
\begin{bmatrix}
* & 0 & 0 & 0 & 0 & * & 0 & 0 & * & 0 & 0 \\
0 & 0 & 0 & * & 0 & 0 & 0 & 0 & 0 & * & * \\
* & 0 & * & * & 0 & * & 0 & 0 & \color{red}{\textbf{+}} & 0 & * \\
* & 0 & 0 & 0 & 0 & * & 0 & 0 & \color{red}{\textbf{+}} & * & 0 \\
0 & * & 0 & 0 & 0 & * & 0 & 0 & 0 & 0 & * \\
* & 0 & 0 & 0 & 0 & \color{red}{\textbf{+}} & * & * & \color{red}{\textbf{+}} & 0 & 0 \\
0 & 0 & 0 & * & * & 0 & 0 & 0 & * & 0 & * \\
* & 0 & 0 & 0 & 0 & \color{red}{\textbf{+}} & 0 & 0 & \color{red}{\textbf{+}} & * & 0 \\
* & 0 & 0 & * & 0 & \color{red}{\textbf{+}} & 0 & * & \color{red}{\textbf{+}} & 0 & 0 \\
0 & 0 & 0 & 0 & 0 & 0 & * & * & 0 & 0 & * \\
* & * & * & 0 & * & \color{red}{\textbf{+}} & 0 & 0 & \color{red}{\textbf{+}} & 0 & 0 \\
\end{bmatrix}
$$
The location of the fill-in elements is: $(6,6)$, $(8,6)$, $(9,6)$, $(11,6)$, $(3,9)$, $(4,9)$, $(6,9)$, $(8,9)$, $(9,9)$, and $(11,9)$.

\item In order to determine the number of fill-in elements with pivot element via Markowitz criterion, we first need to specify the candidate pivot elements. Since we have parameter $\alpha = 0.1$ and

$$
\min_{i,l : u_{i,l}\neq 0} |u_{il}| \geq \frac{1}{7} \max_{i,l}|u_{il}|
$$
Then any nonzero entry in $U^{(k)}$ is a candidate pivot. Now, we need to apply Markowitz criterion i.e., swap the rows and columns with least number of nonzero entries. Let $c_{l} \text{\ and\ } r_{i} \in \mathbb{R}^{11}$ be the two vectors that define the number of nonzero entry per each column and row respectively. 

$$
\setcounter{MaxMatrixCols}{50}
c_{l} = 
\begin{bmatrix}
 7&
 2&
 2&
 4&
 2&
 4&
 2&
 3&
 2&
 3&
 5&
\end{bmatrix}
$$

$$
\setcounter{MaxMatrixCols}{50}
r_{i} = 
\begin{bmatrix}
 3&
 3&
 5&
 3&
 3&
 3&
 4&
 2&
 3&
 3&
 4&
\end{bmatrix}
$$

Row 8 has the least number of nonzero entry among all rows. Columns 2, 3, 5, 7, and 9 has the lowest number of nonzero entry among all columns. However row 8 coincides with column 2, 3, 5, 7, and 9 in zero entry so non of them can be choose as a pivot element. Thus, we need to choose a row with the second lowest number of nonzero i.e., 3 nonzero entry per row. Rows 1, 2, 4, 5, 6, 9, and 10 satisfy  this condition. Following Markowitz criterion, we chose the row with the lowest index i.e., row 1. It is possible now to choose one of the above columns that has 2 nonzero entry and meets row 1 in a nonzero entry i.e, column 9. Thus, \textbf{the pivot element is (1, 9)} which minimizes $(r_{i} -1)(c_{l}-1)$ to be 2. Thus, the worst case number of fill-in elements is 2. 

To determine the fill-in elements location, we need first to bring the pivot element to the top-left corner by swapping column 1 and 9 and then perform the $k$-th step of sparse LU factorization. The location of the two fill-in is shown in the following matrix. 

$$
U^{(k)} = 
\setcounter{MaxMatrixCols}{11}
\begin{bmatrix}
* & 0 & 0 & 0 & 0 & * & 0 & 0 & * & 0 & 0 \\
0 & 0 & 0 & * & 0 & 0 & 0 & 0 & 0 & * & * \\
0 & 0 & * & * & 0 & * & 0 & 0 & * & 0 & * \\
0 & 0 & 0 & 0 & 0 & * & 0 & 0 & * & * & 0 \\
0 & * & 0 & 0 & 0 & * & 0 & 0 & 0 & 0 & * \\
0 & 0 & 0 & 0 & 0 & 0 & * & * & * & 0 & 0 \\
* & 0 & 0 & * & * & \color{red}{\textbf{+}} & 0 & 0 & \color{red}{\textbf{+}} & 0 & * \\
0 & 0 & 0 & 0 & 0 & 0 & 0 & 0 & * & * & 0 \\
0 & 0 & 0 & * & 0 & 0 & 0 & * & * & 0 & 0 \\
0 & 0 & 0 & 0 & 0 & 0 & * & * & 0 & 0 & * \\
0 & * & * & 0 & * & 0 & 0 & 0 & * & 0 & 0 \\
\end{bmatrix}
$$

\end{enumerate}



\section*{Problem 1:}
\begin{enumerate}
\item

\end{enumerate}
\section*{Problem 4:}
We wrote the function \texttt{zkViaLanczos} which computes $Z_k$	given $T_{k}$, $s$ and $s_{0}$. $T_{k}$ is computed from our previous implementation of the nonsymmetric Lanczos in Homework 3 which feed in with the efficient implementation of the $Mv$ and $M^{T}v$ from Problem 2. 

\paragraph{System Specs:} All our experiments run on Intel(R) Xeon(R) CPU E3-1280 v5 with 3.70 GHz and 32 GB of RAM on 64-bit operating system running Windows 7. 

\paragraph{$s_0$ with fast convergence:} We test our implementation of the textbook and Lanczos-based algorithm for different values of $s_{0}$ and found that it runs fairly fast for the small input given in \texttt{FP\_Ex1.mat}; it takes less than a second even for large i.e., $k < 100$. 

We followed the recommendation given in the lectures for how to pick $s_{0}$. We choose $s_{0} = \expnumber{1}{5} + 2\pi i \expnumber{5.5}{8} $. 

\paragraph{Comparison:} We run both our implementation for different values of $k$ and the above $s_{0}$ and compared between both. Table~\ref{tab:comp} should the average different ($\parallel \dot{•} \parallel^{2}$) and the maximum different between the two vectors containing the output of both algorithms for different values of $s$. 

\paragraph{Explanation:} We believe the reason why the textbook algorithm does not perform well is because it depends on computing $M^{j}r$ (to compute the moments) for increasing values of $j$ which convergences quickly to the the eigenvector of $M$ with largest eigenvalue. Thus, the information it contains comes from a single eigenvector where the information should comes from all eigenvectors of $M$. In contrast, Lanczos's $T_{k}$ represents oblique projection of $M$ onto the $K_{k}(M,r)$ Krylov subspace which contains information about $k$ eigenvectors. 









\end{document}
