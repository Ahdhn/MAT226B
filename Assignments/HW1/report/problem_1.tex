\section*{Problem 1:}
\begin{enumerate}
\item Let $A = [a_{j,k}] \in \mathbb{R}^{n\times n} \succ 0$ and $L = [l_{j,k}]$ be its Cholesky factor. Using MATLAB notation, Algorithm \ref{algo:chol} shows Cholesky factorization algorithm

\begin{algorithm}[H]
\SetAlgoLined
\KwIn{$A=[a_{j,k}] \in \mathbb{R}^{n\times n} \succ 0$}
\KwOut{ $L=[l_{j,k}]$ such that $A = LL^{T}$ }
 $l_{j,k} = a_{j,k}, \forall j\geq k \text{\ and\ }  j,k= 1,2, \ldots, n$ \\
 \For{ $k= 1,2, \ldots, n $}{   
  $l_{k,k} = \sqrt{l_{k,k}}$\\
  $l_{k+1:n,k} = \frac{1}{l_{k,k}} l_{k+1:n, k}$\\
  
  \For{$j=k+1, k+2, \ldots, n$}{
	  $l_{j:n,j} = l_{j:n,j} - l_{j:n,k}l_{jk}$\\  
  }
 }
 \caption{Cholesky Factorization}
\label{algo:chol}
\end{algorithm}

\noindent Line 1 in Algorithm \ref{algo:chol} is a memory copy and does not include any flops. Line 3 accounts for $n$ square root operations. On iteration $k$, Line 4 will account for $n-k$ division operations. Since this loop goes from $k= 1,2, \ldots, n $, we get $\sum_{i=1}^{n}(n-k) = \frac{1}{2}n(n-1)$ division operation.

\noindent Line 6 does two operations; subtraction and multiplication, each on a vector of length $(n-j+1)$. Thus, the total cost of the inner loop is 
$$
\sum_{k=1}^{n} \sum_{j=k+1}^{n} 2(n-j+1) = \frac{1}{3}n(n^{2}-1)
$$

Thus, the total cost of Algorithm \ref{algo:chol} is 
$$
n + \frac{1}{2}n(n-1) + \frac{1}{3}n(n^2-1) \text{\ flops}
$$

\item 
Let $A$ be a banded $n\times n$ matrix with bandwidth $2p+1$, i.e., $a_{jk}=0 \text{\ if\ } |j-k|>p$. To show that Cholesky factor $L$ has lower bandwidth $p$, i.e.,  $l_{jk}=0 \text{\ if\ } j-k>p$, we need to show the Cholesky factorization does not introduce any fill-in's. Line 3 and 4 in Algorithm \ref{algo:chol} do not introduce any fill-in's. Line 6 can be re-written (in scalar notation instead of MATLAB's vector notation) as 
$
l_{ik} = l_{ik} - \frac{l_{i,j}}{l_{k,k}} \text{\ and \ } i= j, j+1 \ldots, n
$


 
\item 
\end{enumerate}