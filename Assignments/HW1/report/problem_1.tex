\section*{Problem 1:}
\begin{enumerate}
\item Let $A = [a_{j,k}] \in \mathbb{R}^{n\times n} \succ 0$ and $L = [l_{j,k}]$ be its Cholesky factor. Using MATLAB notation, Algorithm \ref{algo:chol} shows Cholesky factorization algorithm

\begin{algorithm}[H]
\SetAlgoLined
\KwIn{$A=[a_{j,k}] \in \mathbb{R}^{n\times n} \succ 0$}
\KwOut{ $L=[l_{j,k}]$ such that $A = LL^{T}$ }
 $l_{j,k} = a_{j,k}, \forall j\geq k \text{\ and\ }  j,k= 1,2, \ldots, n$ \\
 \For{ $k= 1,2, \ldots, n $}{   
  $l_{k,k} = \sqrt{l_{k,k}}$\\
  $l_{k+1:n,k} = \frac{1}{l_{k,k}} l_{k+1:n, k}$\\
  
  \For{$j=k+1, k+2, \ldots, n$}{
	  $l_{j:n,j} = l_{j:n,j} - l_{j:n,k}l_{jk}$\\  
  }
 }
 \caption{Cholesky Factorization}
\label{algo:chol}
\end{algorithm}

\noindent Line 1 in Algorithm \ref{algo:chol} is a memory copy and does not include any flops. Line 3 accounts for $n$ square root operations. On iteration $k$, Line 4 will account for $n-k$ division operations. Since this loop goes from $k= 1,2, \ldots, n $, we get $\sum_{k=1}^{n}(n-k) = \frac{1}{2}n(n-1)$ division operation.

\noindent Line 6 does two operations; subtraction and multiplication, each on a vector of length $(n-j+1)$. Thus, the total cost of the inner loop is 
$$
\sum_{k=1}^{n} \sum_{j=k+1}^{n} 2(n-j+1) = \frac{1}{3}n(n^{2}-1)
$$

Thus, the total cost of Algorithm \ref{algo:chol} is 
$$
n + \frac{1}{2}n(n-1) + \frac{1}{3}n(n^2-1) \text{\ flops}
$$

\item 
Let $A$ be a banded $n\times n$ matrix with bandwidth $2p+1$, i.e., $a_{jk}=0 \text{\ if\ } |j-k|>p$. To show that Cholesky factor $L$ has lower bandwidth $p$, i.e.,  $l_{jk}=0 \text{\ if\ } j-k>p$, we need to show the Cholesky factorization does not introduce any fill-in's. Line 3 and 4 in Algorithm \ref{algo:chol} do not introduce any fill-in's. 

At step $k$, the factor $l_{jk}$ in Line 6 will be non-zero only for $k \leq j \leq k+p$. Thus, the only possible fill-in's for column $j$ (i.e., at iteration $j$ of the inner loop) is at the first $p$ rows below the diagonal which are already non-zero since $A$ is banded matrix with bandwidth $2p+1$ which means there is $p$ non-zero rows below the diagonal already. 

\item Cholesky factorization can be re-written more efficiently for banded matrices since it is guaranteed to \emph{not} introduce fill-in such that computation can be skipped for the zero elements. Algorithm \ref{algo:chol_band} shows Cholesky factorization algorithm for banded matrices 

\begin{algorithm}[H]
\SetAlgoLined
\KwIn{$A=[a_{j,k}] \in \mathbb{R}^{n\times n} \succ 0 \text{\ with\ bandwidth\ } 2p+1$}
\KwOut{ $L=[l_{j,k}]$ such that $A = LL^{T}$ }
 $l_{j,k} = a_{j,k}, \forall j\geq k \text{\ and\ }  j,k= 1,2, \ldots, n$ \\
 \For{ $k= 1,2, \ldots, n $}{   
  $l_{k,k} = \sqrt{l_{k,k}}$\\
  $l_{k+1:k+p,k} = \frac{1}{l_{k,k}} l_{k+1:k+p, k}$\\
  
   \For{$j=k+1, k+2, \ldots, k+p$}{
	  $l_{j:k+p,j} = l_{j:k+p,j} - l_{j:k+p,k}\quad l_{j,k}$\\  
  }
%\For {$j=k+1, k+2 \ldots, k+p$}{
%		\For {$i=j, j+1, \ldots, p+k$}{
%			$l_{i,j} = l_{i,j} - l_{i,k}l_{j,k}$
%		}
%	}
}
 \caption{Cholesky Factorization for Banded Matrices}
\label{algo:chol_band}
\end{algorithm}


In the algorithm above, we note the following 
\begin{itemize}
\item Line 4 now only operates on the first $p$ rows below the diagonal.% elements of the column $k$.
\item  Line 5 only goes through the first $p$ columns after column $k$ (during iteration $k$) since the factor $l_{jk}$ (Line 6) will be zero for $j > k+p$
%\item Line 6 abandons MATLAB notation to easily illustrates that during computation of column $j$, we need only to go through the rows that will be added to non-zero elements of column $k$ which is reduced by one for every iteration (inner loop at Line 6). 
\item Line 6 now only operates up to row $k+p$ since the rows below $k+p$ for column $k$ (i.e., at iteration $k$) will contain zeros. 
\end{itemize}

\newpage
\item Algorithm \ref{algo:chol_band} requires $n$ square root operations. Line 4 requires only $p$ division. Since Line 4 runs for all $k$ values (except for the last $p$ columns but we ignore this corner case assuming $p\ll n$), then the total number of division done by Line 4 is $np$. 

Line 6 costs $p$ subtraction and division (assuming $p\ll n$). Thus the total cost of the whole loop (Line 5-7) is
%$$
%\sum_{k=1}^{n}\quad \sum_{j=k+1}^{k+p}\quad \sum_{i=j}^{p+k} 2 = np(p+1)
%$$

$$
\sum_{k=1}^{n}\quad \sum_{j=k+1}^{k+p} 2p = 2np^{2}
$$

Thus, the total cost of Algorithm \ref{algo:chol_band} is
$$
2np^{2} + np + n = 
$$
%$$
%n(p+1)^{2}
%$$

\end{enumerate}